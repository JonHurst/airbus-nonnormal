\documentclass[a5paper,11pt,titlepage]{article} \usepackage[utf8]{inputenc}
\usepackage[T1]{fontenc}
\usepackage{txfonts}
\usepackage[margin=0.75in]{geometry}

\title{Zen and the Art of Airbus Failure Handling}
\author{Jon Hurst}

\begin{document}
\maketitle

\section{Introduction}

First of all, there is nothing official about this essay. I describe
herein a mental model of failure handling that I find useful and that
may be of use to you, in part or in whole. Your mileage may very
definitely vary.

\section{Model Description}

Picture, if you will, a mountain spring filling a small stream. This
spring represents your cognitive capacity.

A bamboo channel leads off the stream, with a sluice gate to control the
amount of water it drains. This channel represents the amount of
capacity that is being used for the primary tasks of flying of the
aircraft for PF and the monitoring of same by PM. It should \emph{never}
run dry!

Five large bowls sit by the stream. Mine are made of white marble. They
are labeled as:

\begin{enumerate}
\item Diagnosis
\item Containment
\item Strategy
\item Tactics
\item Review
\end{enumerate}

Finally, you have a ladle, which allows you to transfer water from the
stream to the bowls. Mine is made of heavy duty stainless steel. This
ladle represents the triage process.

When faced with a failure, your aim is to finish up with all the bowls
sufficiently full for a successful outcome. During the process of
filling the bowls the bamboo channel must always have sufficient
flow and the bowls must be, moment to moment, in a state such that the
immediate challenges are met.

\section{An example}

It is probably best to start off with an example, and since the EFATO is
the significant failure that we are most familiar with, we will start
with that.

So you are charging down the runway, you pass V$_1$, and you suddenly
find yourself needing a lot of rudder to keep straight. The sluice gate
is wide open, as you are suddenly dealing with unfamiliar handling
occuring in a high workload flight phase. There is just enough in the
stream for a splash into the Diagnosis bowl. This is mainly just a
recognition of symptoms. You diagnose asymmetry and probable loss of
thrust.

Triage leads to the next splash going in the Tactics bowl. You need to
carry out an asymmetric takeoff procedure: slower rotation,
$12\frac{1}{2}^{\circ}$ pitch target then V$_2$ to V$_2$+15 airspeed
target, $15^{\circ}$ bank angle limit until you've thought about it,
TOGA once transitioned to flight mode. Luckily, you were prescient
enough to have mentally rehearsed this exact procedure just before
applying thrust for takeoff, so the splash you can afford is
sufficient. You get safely airborne.

The Strategy bowl mustn't remain dry for long. The first item into the
strategy bowl is always ``protect consciousness'' and the second is
always ``safe flight path''. In this case, the first is not an issue,
but the second very much is. Your requirement is for a flight path free
of granite and CBs, your preference is for lower workload. Your options
are stay on the SID, fly the EOSID or fly an immediate VMC recovery. You
note that you are about to go IMC, so the EOSID it is. Luckily, this is
a STD EOSID, so you have a bit of breathing space; you just need to
climb straight ahead until you have accelerated to green dot. You tell
PM that this is the plan, and ask him to ``Pull Heading and set a
heading to maintain runway track''.

Next triage item is to do something about the hideous amount of
cognitive flow being diverted down the flying/monitoring channel. A
trimmed aircraft always requires less flow than an untrimmed one, so
trim out the rudder. An autopilot, if available, can bring the required
flow down to a steady trickle, so attempt to engage one. Praise be to
Airbus, it engages and appears to handle the asymmetry. Suddenly there
is plentiful water in the stream.

The strategy could do with a little attention at this point, in that the
knowledge of the plan is confined to the flight deck, and it would be
worth bringing ATC and cabin crew at least a little bit into the loop. A
proactive PM may already have gotten out a Mayday with details of
initial tracking plans. A really proactive PM may have got out an
``Attention Crew at Stations''. If not, it is only a couple of seconds
effort to attend to these details, and it will avoid potentially
dangerous interruptions during the critical steps ahead.

So far, we've just been treating symptoms; we need to add a good ladle
full to the Diagnosis bowl to make any further progress. Airbus advises
leaving this until 400ft aal, but since you applied TOGA nice and early,
you are well above that. You ask the PM to ``Confirm the failure''. If
you are lucky, you will get back an enumeration of the all the symptoms
taking data from all readily available sources, followed by a suggested
diagnosis. If you are less lucky, you may just get the ECAM title, in
which case you will have to do the enumeration yourself. Either way, by
the end of this process you have agreed a preliminary diagnosis across
the flight deck.

Triage now suggests that it is the Containment bowl that needs
attention. You would like that failed engine secured. There are no OEBs
that affect, and the first ECAM procedure covers your aim, so you take
the radios and ask for the ECAM actions. This is a very high risk
time. If you should shut down the wrong engine at this point in time it
may well be your last ever mistake. You successfully secure the engine
and ``Clear Eng''. The Containment bowl is now sufficiently full for a
bit.

What now? A splash to the Tactical bowl gives a requirement to reduce
from TOGA to MCT within 10 minutes, and that is likely beginning to be
an issue. You need to be clean before you can reduce the thrust, so you
need to accelerate to green dot and clean up. You would also like to add
something a bit more substantive to the Strategy bowl. You decide the
priority is a good ladle-full to the Tactical bowl; you request ``Stop
ECAM'' and push the VS knob. The FMGC has detected the engine out
condition as designed, which is nice, and thus the de-activation of SRS
mode automatically advances the speed bug, deleting any restrictions you
may have entered. You clean up and commence an MCT green dot climb.

Time to add a good dollop to the Strategy bowl then. Your requirement of
``safe flightpath'' is still ongoing, but you now introduce a preference
to be within gliding distance of a suitable runway. Options are to route
to the EOSID waypoint, or to a suitable alternative position. You note
that you are above MSA. Being extremely cynical, you discussed during
the pre-takeoff brief that the EOSID waypoint was 25 miles away from the
airport in the wrong direction, and you declared a preferred holding
location at a downwind position. The weather wasn't too bad when you
left, so you advise ATC that you are routing to this position to hold.

Back to the Containment bowl. ``Continue ECAM'', considering OEBS and
requesting ``ECAM Actions'' for each sane ECAM presented and making
disparaging remarks about the start valve fault. All the other bowls are
now sufficiently full for immediate requirements, so we fill the
Containment bowl to the brim, AKA ``ECAM actions complete''.

Now it is time to finish filling the Strategy bowl. Requirements are
``safe flightpath'' and ``LAND ASAP amber''. Preferences are for a
proximate, familiar airport with simple procedures, conservatively long
runways and reasonable weather without icing conditions. With PM you
gather data and generate options, and designate the airport 4 miles away
as plan A and a nearby alternative with better weather as plan B. Now
you have a firm plan, you need to get it out of the flight deck so that
everyone else can get on with sorting out what they need to sort
out. You inform ATC of your intentions in the form of a NITS brief, get
``SCCM to the flight deck'' and give them a NITS brief and then do the
passenger PA, which will basically be… a NITS brief. While the SCCM is
in the flight deck, you take the opportunity to add a splash more to the
diagnosis, by discussing anything they have seen in the cabin.

We are almost done. Diagnosis is as good as it can be. Containment is
sorted. Strategy is in place. We just need to attend to Tactics and we
can get ourselves safely on the ground. Before we do that, however, the
Review bowl needs attention. Review is the difficult process of visiting
each of the other bowls critically in attempt to trap mistakes and
omissions. It needs to be done as methodically and dispassionately as
possible, because you're human and the possibility you shut down the
wrong engine is not a comfortable place to visit.

That done you fill the Tactics bowl by developing a tactical queue. This
is the pre-planning of the rest of the flight determining what abnormal
procedures and techniques you are going to insert into the normal flow,
and when you are going to do it. In this case, you have modified
anti-ice procedures, fuel balancing procedures, modifications to
windshear/EGPWS procedures, some minor limitations on autopilot modes,
asymmetric approach and landing and asymetric go-around to
consider. You set up appropriately, brief, fly a text-book approach and
landing and head to the bar.

\section{Discussion of model components}

\subsection{Flying and monitoring}

History has shown that it is perfectly possible to get so involved in
the handling of a failure that the flight path is fatally
disregarded. Sufficient congnitive capacity must always be devoted to
the flying task by PF and to the monitoring task by PM. Active awareness
of how difficult it is to do this in the face of a complex failure is
the only real defense.

Since the flying/monitoring task is taking resource from a finite pool,
any steps that reduce the amount of capacity required for sufficiency
are helpful.

The main step that can be taken is to grab any safe opportunity to train
manual handling skills. It is an unfortunate fact that the more severe a
failure is, the more chance there is that you will be flying the
aircraft manually, possibly with degraded handling characterstics. If
you have been regularly honing your skills on fine sunny days, the gap
that must be filled when those skills are required on a dark stormy
night will be much smaller, and the amount of spare capacity available
for handling the underlying failure will be much greater. This applies
to monitoring as much as to handling. The automatics are generally so
reliable that atrophied monitoring skills can go unpunished for a very
long time; regularly acting as PM when PF is handling manually provides
a good antidote.

If it is available, use of the autopilot makes an enormous
difference. In particular, monitoring the autopilot flying a hold in VMC
requires a mere trickle of cognitive capacity; IMC when above MSA
requires a little more, but is still a good place to be. If this hold
happens to be at the end of the downwind leg to a suitable runway, then
a lot of your problems are already solved.

If the autopilot is not available, there is often a significant
asymmetry between the capacity drain for handling and the capacity drain
for monitoring, particularly if the handling is being done well. This
can easily lead to PF not having the capacity to properly monitor PM's
actions or properly absorb the intricacies of the failure. Temporarily
swapping control is useful in these circumstances, giving PF the
capacity to review what has been done and expand their situational
awareness bubble.

\subsection{Triage}
The triage process is visualised as a ladle. This is related to the
concept of chunking of cognitive capacity in the face of a requirement
to simultaneously carry out multiple novel tasks.

Human beings are not particularly good at real multi-tasking; we
actually rely on motor programs and rapid task switching. If a task is
novel, we will not have a suitable motor program unless we can re-task
something that is not novel\
\footnote{Re-tasking motor programs is often a good technique. For
  example, for a go-around everything after reading the FMAs is the same
  as a flap 3 takeoff; it is worth practicing a flap 3 takeoff from time
  to time so you can have the motor program available for the
  go-around.}. Interrupting a task before it is complete often leads to
errors, so switching tasks too rapidly should be avoided. Equally, not
switching tasks rapidly enough leads to ``no-one flying the plane''
scenarios. The trick, then, is to methodically break workflow down into
short, focused, completable chunks, each to be applied at the most
opportune moment. The moment to moment selection of the optimum chunk to
apply is the process of triage.

The efficient application of triage relies heavily on having a suitably
stocked ``bag of tricks''. There is, unfortunately, no shortcut to
developing this. It comes from systematic study of the abnormal
procedures and techniques set down by Airbus in the QRH, FCTM and FCOM,
carried out with an eye for commonalities and adaptions of normal
handling techniques. It is not necessary (and likely not possible) to
memorise every checklist and procedure, but it should be possible to
give a quick summary of what you would likely do for any given symptom
and use the checklists and ECAM to fill in the gaps.

\subsection{Diagnosis}

\subsection{Containment}
\subsection{Strategy}

Strategy is the big picture stuff. It is strongly informed by the
diagnosis and effectiveness of containment.

Strategic planning generally rises through three levels, with increasing
amount of cognitive capacity required for each:

\begin{enumerate}
\item Establish a safe flight path
\item Head in a sensible direction
\item Agree destination and fallback position
\end{enumerate}

Each stage of strategic planning can be broken down into four
necessarily sequential steps:

\begin{enumerate}
\item Declare requirements and preferences
\item Generate options
\item Choose a primary and fallback option
\item Communicate the plan
\end{enumerate}

A discussion and clear declaration of requirements and preferences is a
great barrier to rushed and faulty strategic decision making.

The usefulness of the MEL in requirement generation should not be
underestimated. If Airbus allow you to dispatch with something, then,
provided you have the fuel and any MEL conditions for that dispatch are
met, continuing to scheduled destination is probably the right move. The
MEL can also generate ideas for additional containment and tactical
procedures. A classic example of this is the N/W STRG FAULT, which
becomes a lot less frightening once you realise that you can (just)
dispatch with it, and that there are a heap of good suggestions in
associated MEL OPS procedure.

On the flip side, if the MEL forbids dispatch with a failure, a decision
to divert should be strongly considered. An example of this is the
relatively common Blue System Electric Pump failure which feels fairly
minor (mainly just loss of a spoiler per side and slow slats, with the
RAT available if another hydraulic system is lost), yet Airbus won't
let you dispatch with.

There are usually three sets of people outside the cockpit that need to
know your plans: ATC, the cabin crew and the passengers. Under stress,
non-standard communication can easily become muddled. A good tool to
ensure structure is the NITS brief, which can be used to clearly
disseminate the plan to all three groups:

\begin{itemize}
\item \textbf{N}ature of failure
\item \textbf{I}ntentions
\item \textbf{T}ime
\item \textbf{S}pecial requirements
\end{itemize}

\subsection{Tactics}
\subsection{Review}
\end{document}
