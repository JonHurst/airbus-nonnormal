\documentclass[a5paper,11pt,titlepage]{article} \usepackage[utf8]{inputenc}
\usepackage[T1]{fontenc}
\usepackage{txfonts}
\usepackage[margin=0.75in]{geometry}
\usepackage{tcolorbox}

\title{Zen and the Art of Airbus Failure Handling}
\author{Jon Hurst}

\begin{document}
\maketitle

\begin{tcolorbox}[colframe=red]
\section*{Disclaimer}

First of all, there is nothing official about this essay. I describe
herein a mental model of failure handling that I find useful and that
may be of use to you, in part or in whole. Your mileage may very
definitely vary.

\end{tcolorbox}

\section{Model Description}

Picture, if you will, a mountain spring filling a small stream. This
spring represents your cognitive capacity.

A bamboo channel leads off the stream, with a sluice gate to control the
amount of water it drains. This channel represents the amount of
capacity that is being used for the primary tasks of flying of the
aircraft for PF and the monitoring of same by PM. It should \emph{never}
run dry!

Five large bowls sit by the stream. Mine are made of white marble. They
are labeled as:

\begin{enumerate}
\item Diagnosis
\item Containment
\item Strategy
\item Tactics
\item Review
\end{enumerate}

Finally, you have a ladle, which allows you to transfer water from the
stream to the bowls. Mine is made of heavy duty stainless steel. This
ladle represents the triage process.

When faced with a failure, your aim is to finish up with all the bowls
sufficiently full for a successful outcome. During the process of
filling the bowls the bamboo channel must always have sufficient
flow and the bowls must be, moment to moment, in a state such that the
immediate challenges are met.

\section{An example}

It is probably best to start off with an example, and since the EFATO is
the significant failure that we are most familiar with, we will start
with that.

So you are charging down the runway, you pass V$_1$, and you suddenly
find yourself needing a lot of rudder to keep straight. PM remarks
``engine one fail''. The sluice gate is wide open, as you are suddenly
dealing with non-standard handling occuring in a high workload flight
phase. There is just enough in the stream for a splash into the
Diagnosis bowl. This is mainly just a recognition of symptoms. You
diagnose asymmetry and probable loss of thrust.

Triage leads to the next splash going in the Tactics bowl. You need to
carry out an asymmetric takeoff procedure: slower rotation,
$12\frac{1}{2}^{\circ}$ pitch target then V$_2$ to V$_2$+15 airspeed
target, $15^{\circ}$ bank angle limit until you've thought about it,
TOGA once transitioned to flight mode. Luckily, you were prescient
enough to have mentally rehearsed this exact procedure just before
applying thrust for takeoff, so the splash you can afford is
sufficient. You get safely airborne.

The Strategy bowl mustn't remain dry for long: your requirement is
``safe flight path'', your preference to minimise workload. Your options
are stay on the SID, fly the EOSID or fly an immediate VMC recovery. You
note that you are about to go IMC, so the EOSID it is. Luckily, this is
a STD EOSID, so you have a bit of breathing space; you just need to
climb straight ahead until you have accelerated to green dot. You tell
PM that this is the plan, and ask him to ``Pull Heading''. PM does so,
puts out a quick Mayday to tell ATC about the track deviation, asks them
to standby then makes an ``Attention Crew at Stations'' PA.

Next triage item is a splash in the Tactics bowl to do something about
the amount of cognitive flow being diverted down the flying and
monitoring channels. You trim out the rudder and engage an
autopilot. Suddenly there is plentiful water in the cognitive stream.

So far, you've just been treating the worst symptoms; now that more
capacity is available, you add a good ladleful to the Diagnosis bowl.
You ask the PM to ``Confirm the failure''. PM runs through the
indications from the system pages and panels and offers up a diagnosis of
``engine 1 failure with damage''; you concur and tell them so.

Triage now suggests that getting the engine secured is the highest
priority item. You decide to add a ladleful to the Containment
bowl. There are no OEBs that affect, and the first ECAM procedure will
secure the engine, so you take the radios and ask for the ECAM
actions. You successfully secure the engine and ``Clear Eng''. The
Containment bowl is now sufficiently full for a bit.

What now? The highest priority item is the 10 minute TOGA limit. You
need to be clean before you can reduce the thrust, so you add another
ladleful to the Tactical bowl and apply the level acceleration and
clean up technique. You push the VS knob and note that the speed bug
jumps to 250kt as designed. As S speed, you retract the flaps, then at
green dot you start an MCT climb, noting that the speed bug jumps back
to green dot.

Your strategic plan called for a turn towards the holding point at green
dot at this point, but you are now above MSA and the holding point is 25
miles out to sea. You bring the Strategy bowl up to the next level. Your
add the requirement to head towards somewhere sensible, which is either
going to be towards your departure aerodrome or towards the departure
alternate. The weather was OK when you left, so you choose to head to a
late downwind position at the departure aerodrome to hold, and request
vectors from ATC.

Back to the Containment bowl. The other bowls are sufficiently full for
immediate challenges, so you request ``Continue ECAM'', consider OEBS
and request ``ECAM Actions'' for each sane ECAM presented, then continue
standard ECAM procedures until ``ECAM actions complete''.

Now it is time to finish filling the Strategy bowl. Requirements are a
suitable destination and, if possible, a fallback option. Preferences
are for a proximate, familiar airport with simple procedures,
conservatively long runways and reasonable weather without icing
conditions. With PM you gather data, carry out required calculations and
generate options. From these options you designate the departure airport
as your primary and the takeoff alternate as your fallback. You use the
NITS format to inform ATC of your plan, then request the ``SCCM to the
flight deck''. You add a final splash to the diagnosis by getting a
report of symptoms detected from the cabin, then inform the SCCM of the
plan, again in the form of a NITS brief. Finally, you make a quick,
reassuring PA to the passengers.

Now it is time for the Review bowl. You critically visit each of the
other bowls in turn, methodically checking for correctness of actions
and absence of omissions.

That done, you finish filling up the Tactics bowl. You plan modified
anti-ice procedures, fuel balancing procedures, modifications to
windshear/EGPWS procedures, asymmetric approach and landing and
asymmetric go-around. You set up appropriately, brief and fly a text-book
single-engine approach and landing.

You head to the bar.

\section{Discussion of model components}

\subsection{Flying and monitoring}

History has shown that it is perfectly possible to get so involved in
the handling of a failure that the flight path is fatally
disregarded. Sufficient cognitive capacity must always be devoted to
the flying task by PF and to the monitoring task by PM. Active awareness
of how difficult it is to do this in the face of a complex failure is
the only real defence.

Since the flying/monitoring task is taking resource from a finite pool,
any steps that reduce the amount of capacity required for sufficiency
are helpful.

The main step that can be taken is to grab any safe opportunity to train
manual handling skills. It is an unfortunate fact that the more severe a
failure is, the more chance there is that you will be flying the
aircraft manually, possibly with degraded handling characteristics. If
you have been regularly honing your skills on fine sunny days, the gap
that must be filled when those skills are required on a dark stormy
night will be much smaller, and the amount of spare capacity available
for handling the underlying failure will be much greater. This applies
to monitoring as much as to handling. The automatics are generally so
reliable that atrophied monitoring skills can go unpunished for a very
long time; regularly acting as PM when PF is handling manually provides
a good antidote.

If it is available, use of the autopilot makes an enormous
difference. In particular, monitoring the autopilot flying a hold in VMC
requires a mere trickle of cognitive capacity; IMC when above MSA
requires a little more, but is still a good place to be. If this hold
happens to be at the end of the downwind leg to a suitable runway, then
a lot of your problems are already solved.

If the autopilot is not available, there is often a significant
asymmetry between the capacity drain for handling and the capacity drain
for monitoring, particularly if the handling is being done well. This
can easily lead to PF not having the capacity to properly monitor PM's
actions or properly absorb the intricacies of the failure. Temporarily
swapping control is useful in these circumstances, giving PF the
capacity to review what has been done and expand their situational
awareness bubble.

\subsection{Triage}
The triage process is visualised as a ladle. This is related to the
concept of chunking of cognitive capacity in the face of a requirement
to simultaneously carry out multiple novel tasks.

Human beings are not particularly good at real multi-tasking; we
actually rely on motor programs and rapid task switching. If a task is
novel, we will not have a suitable motor program unless we can re-task
something that is not novel\
\footnote{Re-tasking motor programs is often a good technique. For
  example, for a go-around everything after reading the FMAs is the same
  as a flap 3 takeoff; it is worth practising a flap 3 takeoff from time
  to time so you can have the motor program available for the
  go-around.}. Interrupting a task before it is complete often leads to
errors, so switching tasks too rapidly should be avoided. Equally, not
switching tasks rapidly enough leads to ``no-one flying the plane''
scenarios. The trick, then, is to methodically break workflow down into
short, focused, completable chunks, each to be applied at the most
opportune moment. The moment to moment selection of the optimum chunk to
apply is the process of triage.

The efficient application of triage relies heavily on having a suitably
stocked ``bag of tricks''. There is, unfortunately, no shortcut to
developing this. It comes from systematic study of the abnormal
procedures and techniques set down by Airbus in the QRH, FCTM and FCOM,
carried out with an eye for commonalities and adaptions of normal
handling techniques. It is not necessary (and likely not possible) to
memorise every checklist and procedure, but it should be possible to
give a quick summary of what you would likely do for any given symptom
and use the checklists and ECAM to fill in the gaps.

\subsection{Diagnosis}

\subsection{Containment}
\subsection{Strategy}

Strategy is the big picture stuff. It is strongly informed by the
diagnosis and effectiveness of containment.

Strategic planning generally rises through three levels, with increasing
amount of cognitive capacity required for each:

\begin{enumerate}
\item Establish a safe flight path
\item Head in a sensible direction
\item Agree destination and fallback position
\end{enumerate}

Each stage of strategic planning can be broken down into four
necessarily sequential steps:

\begin{enumerate}
\item Declare requirements and preferences
\item Generate options
\item Choose a primary and fallback option
\item Communicate the plan
\end{enumerate}

A discussion and clear declaration of requirements and preferences is a
great barrier to rushed and faulty strategic decision making.

The usefulness of the MEL in requirement generation should not be
underestimated. If Airbus allow you to dispatch with something, then,
provided you have the fuel and any MEL conditions for that dispatch are
met, continuing to scheduled destination is probably the right move. The
MEL can also generate ideas for additional containment and tactical
procedures. A classic example of this is the N/W STRG FAULT, which
becomes a lot less frightening once you realise that you can (just)
dispatch with it, and that there are a heap of good suggestions in
associated MEL OPS procedure.

On the flip side, if the MEL forbids dispatch with a failure, a decision
to divert should be strongly considered. An example of this is the
relatively common Blue System Electric Pump failure which feels fairly
minor (mainly just loss of a spoiler per side and slow slats, with the
RAT available if another hydraulic system is lost), yet Airbus won't
let you dispatch with.

There are usually three sets of people outside the cockpit that need to
know your plans: ATC, the cabin crew and the passengers. Under stress,
non-standard communication can easily become muddled. A good tool to
ensure structure is the NITS brief, which can be used to clearly
disseminate the plan to all three groups:

\begin{itemize}
\item \textbf{N}ature of failure
\item \textbf{I}ntentions
\item \textbf{T}ime
\item \textbf{S}pecial requirements
\end{itemize}

\subsection{Tactics}
\subsection{Review}
\end{document}
